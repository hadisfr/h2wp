\documentclass[11pt,largemargins]{h2wp}

\newcommand{\eventclass}{ریاضیات گسسته}
\newcommand{\eventtype}{رو به سعادت در نوشتار}
\newcommand{\eventcntrbs}{سودابه محمدهاشمی - کیمیا محمدطاهری - یارا کامکار - بهزاد شایق}

\begin{document}
\maketitle

هر کس که وارد مبحث علم ریاضیات میشود باید توانایی درست خواندن و درک کردن و نوشتن مسائل و اثبات ها را کسب کند.
چیزی که برای ما اهمیت دارد درست نویسی است چرا که درست نوشتن اثبات ها نقش به سزایی در درست خواندن و درک کردن آن ها نیز دارد.

خیلی از اوقات پیش می آید که اثبات موضوعی بسیار منطقی ب نظر می آید ولی در واقع کاملا اشتباه هست. 
به همین خاطر هست که ما نیاز داریم یادبگیریم چگونه اثبات ها را دنبال کنیم.
زمانی که اثبات ها را مرحله به مرحله با استدلال های منطقی و درست پیش ببریم میتوانیم از درست بودن و معتبر بودن نتیجه ای که به دست آورده ایم مطمئن باشیم.
به علاوه این که در نهایت وقتی یاد بگیریم چگونه استدلال های به جا بیاوریم و اثباتمان را کامل کنیم ، به فهم عمیق تری از ریاضیات و مسائل میرسیم.

پایه ی درست نویسی ساده است.
فقط کافی است که دقت کنیم هر جمله ای که استفاده میکنیم یکی از موارد زیر باشد:
۱.فرض مسئله
۲.از جمله ی قبلی به طور شفاف نتیجه شده باشد.
۳.درستی آن قبلا اثبات شده باشد.

در این جزوه ما سعی داریم شما را بیشتر با درست نویسی و نکاتی که در این جهت ما را سوق میدهند آشنا کنیم ( تا در نهایت یاد بگیریم در دیگر مراحل زندگی مان هم با منطق باشیم. )

\chapter*{شمارش}

\question
\question

\question
۶۰ دانشجو در کلاس ریاضیات گسسته حضور دارند.در میان هر ۱۰ نفر از این کلاس , حداقل ۳ نفر نمره مبانی یکسانی دارند. ثابت کنید در این کلاس ۱۵ نفر وجود دارند که نمره مبانی آن ها یکسان است.

\solution
$ \hyperref[sec:1]{\textcolor{capri}{(1.3.N.1)}} $
در نظر میگیریم حداکثر تعداد تکرار از یک نمره ۱۴ عدد است که در این صورت حداقل به ۵ نمره متفاوت نیاز است . 
 $ \hyperref[sec:2]{\textcolor{capri}{(1.3.N.2)}} $
در این صورت باز میتوان گروه ۱۰ تایی را از دانش آموزان انتخاب کرد که حداکثر دو نفر نمره یکسان داشته باشند . پس فرض اولیه غلط بوده و مشخص میشود که لااقل از یکی از نمرات وجود دارد که ۱۵ دانش آموز یا بیشتر 
آن نمره را دارند.
 $ \hyperref[sec:3]{\textcolor{capri}{(1.3.T.3)}}  $


\notes

\Nnote
در پاسخ از برهان خلف استفاده شده ولی از آن نام برده نشده است و باید توجه کنیم فرض خلف را حتما بیان کنیم.
\label{sec:1}

\Nnote
باید اصل لانه   کبوتری که از آن استفاده کرده است را نام میبرد و نحوه استفاده از آن مشخص شود.
\label{sec:2} 

\Tnote
پاسخ کامل نیست. پاسخ درست و کامل در پایین آمده است.
\label{sec:3}


\solution*{پاسخ صحیح}

از برهان خلف استفاده می‌کنیم. فرض خلف : فرض کنید در این کلاس هیچ ۱۵ نفری وجود نداشته باشند که نمره‌ی مبانی آن‌ها یکسان باشد.
در این صورت حداکثر ۱۴ نفر وجود دارند که نمره‌ی یکسان داشته باشند. بنابراین  طبق اصل لانه کبوتری حداقل به اندازه‌ی سقف
$\frac{60}{14}$
یعنی ۵ نمره‌ی متفاوت در کلاس وجود دارد. مسئله را به دو حالت تقسیم می‌کنیم ؛
\begin{enumerate}
    \item 
    اگر پنج نمره‌ی متمایز وجود داشته باشند که از هر کدام ۲ عضو
    (دو نفر در کلاس که آن نمره را دارند)
    وجود داشته باشد؛ در این صورت از هر کدام از این نمرات دو عضو را درنظر گرفته و به مجموعه‌ای ۱۰ عضوی
    می‌رسیم که هیچ سه نفری در آن نمره‌ی یکسان ندارند که این خلاف فرض مسئله است و به تناقض رسیدیم.
    پس فرض خلف رد شده و حداقل ۱۵ نفر وجود دارند که نمره‌ی یکسانی داشته باشند.
    
    \item 
    اگر پنج نمره‌ی متمایز، هرکدام دارای حداقل دو عضو وجود نداشته باشند؛
    در این صورت 
    $k \le 4$
    نمره‌ی متمایز با بیش از یک عضو داریم
    (مجموعه‌ی این نمرات را A بنامیم)
    که با توجه به فرض خلف، حداکثر تعداد
    $14 \times k$
    عضو را پوشش می‌دهند. بنابراین حداقل 
    $60 - 14k$
    نفر باقی مانده که هیچ دو تایی نمی‌توانند دارای نمره‌ی یکسان باشند
    (در غیر این صورت تعداد نمره‌های متمایز دارای بیشتر مساوی ۲ عضو به حداقل k+1 می‌رسد).
    بنابراین هر یک از این اعضا دارای نمره‌ای متمایز است
    (مجموعه‌ی این اعضا را B بنامیم).
    می‌توان با انتخاب دو عضو از هر نمره‌ی مجموعه‌ی
    A
    و تمام اعضای مجموعه‌‌ی
    B
    به مجموعه‌ای متشکل از
    $N = 2k + 60 - 14k = 60 - 12k$
    عضو رسید که
    $k \le 4 \Rightarrow N \ge 12$
    و هیچ سه عضوی در آن دارای نمره‌ی یکسان نیستند.
    هر ده عضوی از این مجموعه انتخاب شود، نقض فرض مسئله است و به تناقض رسیدیم.
    پس فرض خلف رد شده و حداقل ۱۵ نفر وجود دارند که نمره‌ی یکسانی داشته باشند.
\end{enumerate}


\chapter*{شمارش}
\question
\question
\question
ضریب عبارت $ x ^ {12}$
در بسط عبارت
$(1-4x)^{-5}$
را بیابید.

\solution
طبق بسط دوجمله ای داریم:  
$ \hyperref[sec:1]{\textcolor{capri}{(1.4.N.1)}}  $     
    \begin{align*}
    \frac{1}{(1-4x)^5}=\sum\limits_{k=0}^{\infty} \binom{k+4}{k}  4 ^ k  x^ k              
    \end{align*}
    
    جمله ۱۲ ام دنباله 
    $ a_n$
  ضریب 
  $ x ^ {12} $
  است.
$ \hyperref[sec:2]{\textcolor{capri}{(1.4.N.2)}}  $

\begin{align*}
  \longrightarrow a_{12} = \binom{16}{12} 4 ^ {12}
\end{align*}



\notes

\Nnote
بهتر است اصل بسط دوجمله ای هم نوشته شود. \label{sec:1} 

\Nnote
قبل از استفاده از متغیر باید آن را تعریف کرد. تعریف دنباله $a_n$ ضروری است. \label{sec:2} 		


\solution*{پاسخ صحیح}

طبق جدول
  Functions Generating Useful
از کتاب Rosen
که استاد نیز به آن اشاره کردند داریم:\\
$$(1 - x) ^ {-n} = \sum_{k = 0}^{\infty} \binom{n + k - 1}{k} x^{k} $$
  بنابراین در این سوال داریم:
$$(1 - 4x) ^ {-5} =
\sum_{k = 0}^{\infty} \binom{5 + k - 1}{k} (4x)^k $$

جمله $x^{12}$ به ازای مقدار $k = 12$ ساخته می‌شود. بنابراین جواب برابر خواهد بود با: 
$$\binom{16}{12} (4)^{12}$$

\end{document}
