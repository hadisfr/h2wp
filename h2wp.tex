\documentclass[11pt,largemargins]{h2wp}

\newcommand{\eventclass}{ریاضیات گسسته}
\newcommand{\eventtype}{رو به سعادت در نوشتار}
\newcommand{\eventcntrbs}{سودابه محمدهاشمی - کیمیا محمدطاهری}

\begin{document}
\maketitle

هر کس که وارد مبحث علم ریاضیات میشود باید توانایی درست خواندن و درک کردن و نوشتن مسائل و اثبات ها را کسب 
چیزی که برای ما اهمیت دارد درست نویسی است چرا که درست نوشتن اثبات ها نقش به سزایی در درست خواندن و درک کردن آن ها نیز دارد.

خیلی از اوقات پیش می آید که اثبات موضوعی بسیار منطقی ب نظر می آید ولی در واقع کاملا اشتباه هست. 
به همین خاطر هست که ما نیاز داریم یادبگیریم چگونه اثبات ها را دنبال کنیم.
زمانی که اثبات ها را مرحله به مرحله با استدلال های منطقی و درست پیش ببریم میتوانیم از درست بودن و معتبر بودن نتیجه ای که به دست آورده ایم مطمئن باشیم.
به علاوه این که در نهایت وقتی یاد بگیریم چگونه استدلال های به جا بیاوریم و اثباتمان را کامل کنیم ، به فهم عمیق تری از ریاضیات و مسائل میرسیم.

پایه ی درست نویسی ساده است.
فقط کافی است که دقت کنیم هر جمله ای که استفاده میکنیم یکی از موارد زیر باشد:
۱.فرض مسئله
۲.از جمله ی قبلی به طور شفاف نتیجه شده باشد.
۳.درستی آن قبلا اثبات شده باشد.

در این جزوه ما سعی داریم شما را بیشتر با درست نویسی و نکاتی که در این جهت ما را سوق میدهند آشنا کنیم ( تا در نهایت یاد بگیریم در دیگر مراحل زندگی مان هم با منطق باشیم. )

\chapter*{شمارش}

\question
\question

\question
۶۰ دانشجو در کلاس ریاضیات گسسته حضور دارند.در میان هر ۱۰ نفر از این کلاس , حداقل ۳ نفر نمره مبانی یکسانی دارند. ثابت کنید در این کلاس ۱۵ نفر وجود دارند که نمره مبانی آن ها یکسان است.

\solution
$ \hyperref[sec:1]{\textcolor{capri}{(1.3.N.1)}} $
در نظر میگیریم حداکثر تعداد تکرار از یک نمره ۱۴ عدد است که در این صورت حداقل به ۵ نمره متفاوت نیاز است . 
 $ \hyperref[sec:2]{\textcolor{capri}{(1.3.N.2)}} $
در این صورت باز میتوان گروه ۱۰ تایی را از دانش آموزان انتخاب کرد که حداکثر دو نفر نمره یکسان داشته باشند . پس فرض اولیه غلط بوده و مشخص میشود که لااقل از یکی از نمرات وجود دارد که ۱۵ دانش آموز یا بیشتر 
آن نمره را دارند.
 $ \hyperref[sec:3]{\textcolor{capri}{(1.3.T.1)}}  $


\notes

\Nnote
در پاسخ از برهان خلف استفاده شده ولی از آن نام برده نشده است و باید توجه کنیم فرض خلف را حتما بیان کنیم.
\label{sec:1}

\Nnote
باید اصل لانه   کبوتری که از آن استفاده کرده است را نام میبرد و نحوه استفاده از آن مشخص شود.
\label{sec:2} 

\Tnote
پاسخ کامل نیست. پاسخ درست و کامل در پایین آمده است.
\label{sec:3}


\solution*{پاسخ صحیح}

از برهان خلف استفاده می‌کنیم. فرض خلف : فرض کنید در این کلاس هیچ ۱۵ نفری وجود نداشته باشند که نمره‌ی مبانی آن‌ها یکسان باشد.
در این صورت حداکثر ۱۴ نفر وجود دارند که نمره‌ی یکسان داشته باشند. بنابراین  طبق اصل لانه کبوتری حداقل به اندازه‌ی سقف
$\frac{60}{14}$
یعنی ۵ نمره‌ی متفاوت در کلاس وجود دارد. مسئله را به دو حالت تقسیم می‌کنیم ؛
\begin{enumerate}
    \item 
    اگر پنج نمره‌ی متمایز وجود داشته باشند که از هر کدام ۲ عضو
    (دو نفر در کلاس که آن نمره را دارند)
    وجود داشته باشد؛ در این صورت از هر کدام از این نمرات دو عضو را درنظر گرفته و به مجموعه‌ای ۱۰ عضوی
    می‌رسیم که هیچ سه نفری در آن نمره‌ی یکسان ندارند که این خلاف فرض مسئله است و به تناقض رسیدیم.
    پس فرض خلف رد شده و حداقل ۱۵ نفر وجود دارند که نمره‌ی یکسانی داشته باشند.
    
    \item 
    اگر پنج نمره‌ی متمایز، هرکدام دارای حداقل دو عضو وجود نداشته باشند؛
    در این صورت 
    $k \le 4$
    نمره‌ی متمایز با بیش از یک عضو داریم
    (مجموعه‌ی این نمرات را A بنامیم)
    که با توجه به فرض خلف، حداکثر تعداد
    $14 \times k$
    عضو را پوشش می‌دهند. بنابراین حداقل 
    $60 - 14k$
    نفر باقی مانده که هیچ دو تایی نمی‌توانند دارای نمره‌ی یکسان باشند
    (در غیر این صورت تعداد نمره‌های متمایز دارای بیشتر مساوی ۲ عضو به حداقل k+1 می‌رسد).
    بنابراین هر یک از این اعضا دارای نمره‌ای متمایز است
    (مجموعه‌ی این اعضا را B بنامیم).
    می‌توان با انتخاب دو عضو از هر نمره‌ی مجموعه‌ی
    A
    و تمام اعضای مجموعه‌‌ی
    B
    به مجموعه‌ای متشکل از
    $N = 2k + 60 - 14k = 60 - 12k$
    عضو رسید که
    $k \le 4 \Rightarrow N \ge 12$
    و هیچ سه عضوی در آن دارای نمره‌ی یکسان نیستند.
    هر ده عضوی از این مجموعه انتخاب شود، نقض فرض مسئله است و به تناقض رسیدیم.
    پس فرض خلف رد شده و حداقل ۱۵ نفر وجود دارند که نمره‌ی یکسانی داشته باشند.
\end{enumerate}


\question
ضریب عبارت $ x ^ {12}$
در بسط عبارت
$(1-4x)^{-5}$
را بیابید.

\solution
طبق بسط دوجمله ای داریم:  
$ \hyperref[sec:1]{\textcolor{capri}{(1.4.N.1)}}  $     
    \begin{align*}
    \frac{1}{(1-4x)^5}=\sum\limits_{k=0}^{\infty} \binom{k+4}{k}  4 ^ k  x^ k              
    \end{align*}
    
    جمله ۱۲ ام دنباله 
    $ a_n$
  ضریب 
  $ x ^ {12} $
  است.
$ \hyperref[sec:2]{\textcolor{capri}{(1.4.N.2)}}  $

\begin{align*}
  \longrightarrow a_{12} = \binom{16}{12} 4 ^ {12}
\end{align*}



\notes

\Nnote
بهتر است اصل بسط دوجمله ای هم نوشته شود. \label{sec:1} 

\Nnote
قبل از استفاده از متغیر باید آن را تعریف کرد. تعریف دنباله $a_n$ ضروری است. \label{sec:2} 		


\solution*{پاسخ صحیح}

طبق جدول
  Functions Generating Useful
از کتاب Rosen
که استاد نیز به آن اشاره کردند داریم:\\
$$(1 - x) ^ {-n} = \sum_{k = 0}^{\infty} \binom{n + k - 1}{k} x^{k} $$
  بنابراین در این سوال داریم:
$$(1 - 4x) ^ {-5} =
\sum_{k = 0}^{\infty} \binom{5 + k - 1}{k} (4x)^k $$

جمله $x^{12}$ به ازای مقدار $k = 12$ ساخته می‌شود. بنابراین جواب برابر خواهد بود با: 
$$\binom{16}{12} (4)^{12}$$

\question

\question 
   با استفاده از توابع مولد نشان دهید تعداد روش های انتخاب ۴ عضو دو به دو  نامتوالی از مجموعه اعداد n
			 ,...,1,2,3
			 برابر با 
			 انتخاب 4 از n-3 است.
			 
\solution
یک زیرمجموعه از این نوع مثلا {۱و۳و۷و۱۰} را انتخاب و نابرابری های اکید 
    $ \\ 0 < 1 < 3 < 7 < 10 < n+1 \\ $ 
   را در نظر میگیریم. و بررسی میکنیم چند عدد صحیح بین هر دو عدد متوالی از این اعداد وجود دارند. در اینجا ۰ و ۱ و۳ و۲ و n-۱۰ 
   را به دست می آوریم: ۰ زیرا عددی صحیح بین ۰ و ۱ وجود ندارد و ۱ زیرا تنها عدد ۲ بین ۱ و۳ وجود دارد و ۳ زیرا اعداد صحیح ۴ و۵ و۶ بین ۳ و۷ وجود دارند و ... .
   مجموع این ۵ عدد صحیح برابر
   $ 0 + 1 + 3 + 2 + n-10 = n-4 $ 
     است.
     $ \hyperref[sec:1]{\textcolor{capri}{(1.6.E.1)}}  $
     
    پس تابع مولد زیر را داریم.
    
   \begin{align*}
   G(x) = ( 1 + x^2 + x^3 + ...)^2 (x + x^2 + x^3 + ...)^3 = ( \sum_{k = 0}^{\infty} x^k)^2 ( \sum_{k = 0}^{\infty} x^{k+1})^3 =\\ \frac{1}{(1-x)^2} . (\frac{x}{1-x})^3 = \frac{x^3 }{(1-x)^5} = x^2 (1-x)^{-5} = x^3 \sum_{k=0}^{\infty} \binom{k+4}{k} x ^ k =\\ \sum_{k=0}^{\infty} \binom{k+4}{k} x^{k+3} = \hyperref[sec:2]{\textcolor{capri}{(1.6.E.2)}}  \hspace{0.5 cm} \sum_{k=0}^{\infty}\binom{k+1}{k-3} x^k
\end{align*}    
      
      به دنبال ضریب
      $ x ^ {n-4 } $
      میگشتیم پس 
      $ k = n-4 $
      و جواب نهایی برابر است با
      $\binom{n-3}{n-7} = \binom{n-3}{4} \\ \\ \hyperref[sec:3]{\textcolor{capri}{(1.6.N.1)}} \\   $
  
  
      
\notes

\Enote
  مثال زدن باید به صورتی باشد که حذف آن اختلالی در فهم جواب ایجاد نکند . در اینجا اگر مثال پاراگراف اول را حذف کنیم مشخص نیست تابع مولد برچه اساسی نوشته شده است. پس باید توضیحی درمورد تابع مولد و جمله ای که به دنبال ضریب آن هستیم بدهیم \label{sec:1} 
		
\Enote
     نیاز هست که کاملا گفته شود چه تغییر متغیری انجام میشود . در اینجا تغییر متغیر $ k \rightarrow {k+3} $   را داریم.  همیشه به هنگام تغییر متغیر توجه کنیم ممکن است کران ها تغییر کنند. در اینجا کران پایین از صفر به سه میرود. \label{sec:2} 
	 صورت اصلاح شده:
	 $ \sum_{k=3}^{\infty}\binom{k+1}{k-3} x^k $
	
\Nnote		
	  در طی پاسخ به سوال خوب است دقت کنیم همه ی اعداد را یا فارسی یا انگلیسی 
	  بنویسیم. \label{sec:3} 
	  
	  
\solution
  تابع مولد فاصله از مبدا:
   \begin{align*}
   G(x)=(1+x+x^2+...)(x+x^2+x^3+...)^3
   \end{align*}
   در مجموع n-4 عدد داریم . توان های x باید بین مبدا و مقصد باشند پس باید توانی از x را که کوچک تر یا مساوی n-4 هستند را بیابیم:
   
   \begin{align*}
   G(x)= \frac{x^3}{(1-x)^4} = x^3 (1-x)^{-4} = x^3 \sum_{k=0}^{\infty}\binom{k+3}{3} x^k \\ 
   \longrightarrow \sum_{k=0}^{\infty}\binom{x+k}{k} = \frac{1}{(1+x)^{k+1}} 
   \longrightarrow k+3 \le n-4 \rightarrow k \le n-7
   \end{align*}
   
   \begin{align*}
     \binom{n+1}{r+1} = \sum_{k=r}^{n} \binom{k}{r} \hspace{2cm} (1) \\
  \end{align*}
  
   مجموع حالات: 
   \begin{align*}
   \longrightarrow \sum_{k=0}^{n-7}\binom{k+3}{3} \xrightarrow{(1)}  \binom{n-7+4}{4} = \binom{n-3}{4}   \hspace{2cm} 
    { \hyperref[sec:4]{\textcolor{capri}{(1.6.E.3)}}}
   \end{align*}
   
   
   
 \notes
 
 \Enote
   به هنگام جایگذاری در فرمول باید جایگذاری ها واضح باشد. در این مثال در فرمول (۱) کران پایین از r هست ولی در قسمتی که از آن استفاده شده کران پایین از ۰ است. همین مطلب گویای آن است که به توضیحات بیشتری نیاز هست. \label{sec:4} 
		
		عبارت زیر صورت کامل شده این نکته است:
		\begin{align*}
		\longrightarrow \sum_{k=0}^{n-7} \binom{k+3}{k} = \sum_{k=3}^{k-4} \binom{k}{k-3} = \sum_{k=3}^{n-4} \binom{k}{3} \xrightarrow[{r \rightarrow 3},{ n \rightarrow {n-4} }]{(1)} \binom{n-3}{4}
\end{align*}	


\solution*{پاسخ صحیح}

   تعداد عضو‌های  انتخاب نشده کوچکتر از عضو اول انتخاب شده را 
                $x_1$،
                عضوهای انتخاب نشده بین عضو اول و دوم انتخاب شده را
                $x_2$،
                عضو‌های انتخاب نشده بین عضو دوم و سوم انتخاب شده را
                $x_3$،
                عضو‌های انتخاب نشده بین عضو سوم و چهارم انتخاب شده را
                $x_4$ و
                عضوهای انتخاب نشده بزرگ‌تر از چهارمین عضو انتخاب شده را 
                $x_5$
                می‌گیریم. کافی است تعداد جواب‌های صحیح نامنفی معادله زیر را با شرایط 
                $x_1, x_5 \geq 0 \; x_2 x_3, x_4 \geq 1$
                بشماریم
               
                $$x_1 + x_2 + x_3 + x_4 + x_5 = n - 4$$
                که برابر است با ضریب
                $x^{n - 4}$
                در عبارت:\\
              
                $$(1 + x + x^2 + ...)(x + x^2 + x^3 + ...)(x + x^2 + x^3 + ...)(x + x^2 + x^3 + ...)(1 + x + x^2 + ...) = \frac{x^3}{(1 - x)^5}$$
                
                 بنابراین کافی است ضریب $x^{n - 7}$ را در بسط
                  $(1 - x)^{-5}$
         بشماریم . 
               
               
                   طبق جدول  
                  Functions Generating Useful
		از کتاب Rosen
		که استاد نیز به آن اشاره کردند داریم:\\
       
		    $$(1 - x) ^ {-n} = \sum_{k = 0}^{\infty} \binom{n + k - 1}{k} x^{k} $$

	    بنابراین در این سوال داریم:
                
                    $$(1 - x)^{-5} = \sum_{k = 0}^{\infty} \binom{5 + k - 1}{k}x^k
                    = \sum_{k = 0}^{\infty} \binom{k + 4}{k}x^k
                    = \sum_{k = 0}^{\infty} \binom{k + 4}{4}x^k$$
             
                $x^{n - 7}$ به ازای $k = n -7$ ساخته می‌شود. بنابراین جواب برابر است با:\\
               
                    $$\binom{n - 3}{4}$$
        

\question
 اتحاد زیر را ثابت کنید.
     \begin{align*}
     \sum_{i=0}^{n} i{\binom{n}{i}}^2 = \frac{n}{2} \binom{2n}{n}
     \end{align*}
     
\solution
\begin{align*}
A=\sum_{i=0}^{n} i{\binom{n}{i}}^2 \xrightarrow{\times 2} 2A = \sum_{i=0}^{n} i{\binom{n}{i}}^2 + \sum_{i=0}^{n} i{\binom{n}{i}}^2 \xrightarrow{ j=n-i } \\ 
2A = \sum_{i=0}^{n} i{\binom{n}{i}}^2 + \sum_{j=0}^{n} (n-j){\binom{n}{n-j}}^2 \rightarrow 2A = \sum_{i=0}^{n} i{\binom{n}{i}}^2 + \sum_{j=0}^{n} (n-j){\binom{n}{j}}^2 = \\
 \sum_{i=0}^{n} (i + (n-i)) {\binom{n}{i}}^2 \rightarrow 2A = n \sum_{i=0}^{n} {\binom{n}{i}}^2 \rightarrow 2A = n \sum_{i=0}^{n} \binom{n}{i} \binom{n}{n-i}
\end{align*}
بنابراین میدانیم که:
  $ \hyperref[sec:1]{\textcolor{capri}{(1.7.N.1)}}  $
  \begin{center}
  $  2A = n\binom{2n}{n} \rightarrow A= \frac{n}{2} \binom{2n}{n} $
  \end{center}


\notes

\Nnote  باید فرمول و اتحاد های مورد استفاده و رفرنس معتبر آن ‌ذکر شود . به عنوان رفرنس اسم اتحاد هم کافی است. 
 \label{sec:1}         	 
 
 
 \solution*{پاسخ صحیح}
 طبق اتحاد واندرموند داریم:

$$ \sum_{i=0}^{k} \binom{m}{i} \binom{n}{k-i} = \binom {m+n}{k} \hspace{2cm} (1) $$
\begin{align*}
A=\sum_{i=0}^{n} i{\binom{n}{i}}^2 \xrightarrow{\times 2} 2A = \sum_{i=0}^{n} i{\binom{n}{i}}^2 + \sum_{i=0}^{n} i{\binom{n}{i}}^2 \xrightarrow{ j=n-i } \\ 
2A = \sum_{i=0}^{n} i{\binom{n}{i}}^2 + \sum_{j=0}^{n} (n-j){\binom{n}{n-j}}^2 \rightarrow 2A = \sum_{i=0}^{n} i{\binom{n}{i}}^2 + \sum_{j=0}^{n} (n-j){\binom{n}{j}}^2 = \\
 \sum_{i=0}^{n} (i + (n-i)) {\binom{n}{i}}^2 \rightarrow 2A = n \sum_{i=0}^{n} {\binom{n}{i}}^2 \rightarrow 2A = n \sum_{i=0}^{n} \binom{n}{i} \binom{n}{n-i} \xrightarrow[m=n=k]{(1)} \\
 2A= n \binom{2n}{n} \rightarrow A= \frac{n}{2} \binom{2n}{n}
\end{align*}


\end{document}
