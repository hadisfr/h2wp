\documentclass[11pt,largemargins]{h2wp}

\newcommand{\eventclass}{ریاضیات گسسته}
\newcommand{\eventtype}{افکار سازگار ، نوشتار نابکار}
\newcommand{\eventcntrbs}{سودابه محمدهاشمی - کیمیا محمدطاهری}

\begin{document}
\maketitle

هرگاه نویسنده‌ای توانایی درست نوشتن را نداشته باشد در انتقال صحیح تراوشات فکری خود به خواننده ناکام می‌ماند. نوشتار در خصوص مباحث علم ریاضیات گسسته نیز علاوه بر درک درست از مفاهیم، نیاز به دانش «صحیح نوشتن» و روش به تحریر درآوردن مسائل و اثبات‌ها دارد تا بتواند هدف انتقال بی‌ کم و کاست به خواننده را کسب نماید.

چیزی که واقعا اهمیت دارد این است که با درست نوشتن، خواننده درک صحیحی از راه و روش حل مسائل و اثبات‌ها كسب كند و توانایی درک و خلاقیت در کشف روش‌های حل مسائل را در خود ارتقاء بخشد.
در بیشتر مواقع اثبات مسائل آسان و منطق حل مسائل بسیار دست یافتنی به نظر می‌آید ولی در واقع در زمان نوشتن حل مسائل و اثبات‌ها رویکرد اشتباهی را پیش می‌گیریم.
به همین خاطر است که ما نیاز داریم یاد بگیریم كه چگونه اثبات‌ها را دنبال کنیم.

هر چند كه رعایت حداقل معیارهای درست نویسی و پیشبرد مرحله به مرحله‌ی حل مسائل و اثبات‌ها با استدلال‌های منطقی، خیال ما را نسبت به درست و معتبر بودن نوشته‌ی خود راحت می‌نماید ولی مطالعه‌ی متون فنی و تعمق در نوشتارهای غنیِ ما را به مرحله‌ای از بلوغ در نویسندگی می‌رساند كه علاوه بر رسیدن به بهترین نتیجه‌ی ممكن، ما را در انتقال صحیح معنا و مفهوم و درك صحیح نوشته بسیار موفق می‌سازد
و در نهایت زمانی که فرا بگیریم، چگونه با اتكا به استدلال‌های درست اثباتمان را کامل کنیم، به فهم عمیق‌تری از مسائل می‌رسیم.

پایه و بنیاد درست نویسی بسیار آسان است. تنها کافی است که دقت نماییم تا جملات استفاده شده به یکی از اشکال زیر باشد:

\begin{enumerate}
\item
خود فرض مسئله باشد.
\item
به صورت كاملا شفاف و واضح از جملات قبلی نتیجه شده باشد.
\item
درستی آن قبلا اثبات شده باشد.
\end{enumerate}

در این جزوه تمام سعی و کوشش ما بر این بوده است تا شما را بیشتر با «درست نویسی» و نکاتی که خواننده را به این جهت سوق دهد آشنا کنیم.
امید است كه فراگیری نكات «درست نویسی» در تمامی مراحل زندگی راهبر و راهنمای شما باشد.


\textbf{انواع نکات}

در این جزوه با سه دسته نکته مواجه هستیم:
\begin{enumerate}

\item
دسته $E$: نکات درست نویسی که رعایت نکردن آن باعث ناقص شدن اثبات و در نتیجه کسر نمره می شود.
\item
دسته $N$: نکات درست نویسی که رعایت کردن آن‌ها واجب نیست, امابه خوانایی راه حل, ابهام زدایی, پرهیز از تکرار و جلوگیری از خطا کمک می کنند.
\item
دسته $T$: دام های آموزشی و خطاهای رایج در حل سوالات.

\end{enumerate}
\chapter*{شمارش}

%% question 1

\question
سه مهرۀ رخ متمایز و صفحه شطرنجی $8\times8$ داریم. به چند روش می‌توان این سه مهره را در سه خانه از این صفحه قرار داد به طوری که حداقل یک مهره وجود داشته باشد که توسط هیچ مهره‌ای تهدید نمی شود؟
\solution
  سوال را با اصل متمم حل می کنیم:\\
      -  کل حالات:\\
        \[64\times63\times62\]
      -  حالات نامطلوب: حالاتی که همه رخ‌ها تهدید بشوند.\\
         رخ اول برای قرار گیری در صفحه شطرنجی 64 حالت دارد, حال چون رخ اول باید تهدید بشود رخ دوم را باید در سطر یا ستون رخ اول قرار بدهیم که ۱۴ حالت دارد. چون رخ سوم هم باید تهدید بشود باید در سطر یا ستون یکی از رخ ها باشد که در مجموع شامل ۶ خانه در سطر یا ستون مشترک دو رخ قبلی و ۱۴ خانه در سطرها یا ستون های غیر مشترک دو رخ است. پس کل حالت ها برابر است با:\\
         \[64\times14\times20\]
-         حالات مطلوب: طبق اصل متمم برابر است با:
         \[64\times63\times62 - 64\times14\times20\]
\notes
\Tnote{n1}
نشمردن همه حالت ها: در اینجا تمام حالات نامطلوب محاسبه نشده است, زیرا این امکان وجود دارد که رخ اول توسط رخ دوم تهدید نشود و این حالت در نظر گرفته نشده است.
\Nnote{n2}
بهتر بود اشاره شود که  به دلیل تمایز رخ ها چنین نتیجه ای گرفته شده است.

\solution
     -  کل حالات:\\
        \[64\times63\times62\]
      -  حالات نامطلوب: حالاتی که همه رخ ها تهدید بشوند.\\
         \[64\times7\times20\times2\]
-         حالات مطلوب: طبق اصل متمم برابر است با:
         \[64\times63\times62 - 64\times14\times20\]
\notes
\Nnote{n3}
نبود توضیحات کافی برای عبارت: به دلیل نبود توضیحات کافی, تشخیص چرایی غلط بودن جواب نهایی ممکن نیست.
\solution*{پاسخ صحیح}

   -  کل حالات: به دلیل تمایز رخ ها برابر است با:\\
        \[ P(64,3) = 64\times63\times62\]
      -  حالات نامطلوب: حالاتی که همه رخ ها تهدید بشوند.\\
      دو حالت داریم:
      \begin{enumerate}
\item
رخ اول رخ دوم را تهدید کند:\\
         رخ اول برای قرار گیری در صفحه شطرنجی 64 حالت دارد, حال چون رخ اول باید توسط رخ دوم تهدید بشود رخ دوم را باید در سطر یا ستون رخ اول قرار بدهیم که ۱۴ حالت دارد. چون رخ سوم هم باید تهدید بشود باید در سطر یا ستون یکی از رخ ها باشد که در مجموع شامل ۶ خانه در سطر یا ستون مشترک دو رخ قبلی و ۱۴ خانه در سطرها یا ستون های غیر مشترک دو رخ است. پس کل حالت ها برابر است با:\\
         \[64\times14\times20\]
\item
رخ اول رخ دوم را تهدید نکند:\\
         رخ اول برای قرار گیری در صفحه شطرنجی 64 حالت دارد, حال چون رخ اول نباید توسط رخ دوم تهدید بشود رخ دوم را در خانه ای به جز سطر یا ستون رخ اول قرار بدهیم که ۴۹ حالت دارد. حال رخ سوم باید هر دو رخ را تهدید کند پس باید در یکی از محل های تقاطع سطر و ستون رخ اول و رخ دوم قرار بگیرد که دو حالت دارد, پس کل حالت ها برابر است با:\\
         \[64\times49\times2\]
         
      \end{enumerate}
-         حالات مطلوب: طبق اصل متمم برابر است با:
         \[64\times63\times62\ - (64\times14\times20\ + 64\times49\times2\ )\]
  
%% question 2
       
\question
  اتحاد زیر را ثابت کنید.\\ \\
     \centerline{$1^2 {n \choose 1} + 2^2 {n \choose 2} + 3^2 {n \choose 3} + ... + n^2 {n \choose n} = n(n+1)2^{n-2}$}\\ \\
\solution
   فرض کنید $ P = \displaystyle\sum_{k=0}^{n} {{k^2}\binom{n}{k}}$ بیانگر تعداد راه های انتخاب یک کمیته از بین $n$ کاندیدا است به طوری که یک فرد یا دو فرد متمایز، رئیس کمیته باشند.  حال این شمارش را به روش دیگری انجام می دهیم.
    \begin{enumerate}
        \item 
        با فرض داشتن یک رئیس، رئیس را انتخاب کرده و تصمیم می‌گیریم که بقیه افراد حضور داشته باشند یا خیر و حالات به دست آمده را جمع می‌کنیم با حالاتی که 2 رئیس را انتخاب کردیم در مورد حضور یا عدم حضور بقیه افراد تصمیم گرفتیم:
        \[P = n\times{2^{n-1}} + n\times(n-1)\times{2^{n-2}} = n\times(n+1)\times{2^{n-2}}\]
    \end{enumerate}
    از تساوی این ۲ حالت حکم مساله اثبات می‌شود:
    \[\displaystyle\sum_{k=0}^{n} {{k^2}\binom{n}{k}} = n\times(n+1)\times{2^{n-2}}\]
    \\ 
\notes
\Enote{n4}
عدم تطابق توضیحات با فرمول نوشته شده, انتخاب دو رئیس از میان $n$ نفر $\binom{n}{2}$ حالت دارد نه $n\times(n-1)$. 
\Nnote{n5}
بهتر است روش اثبات (دوگانه شماری)ذکر شود.
\Enote{n6}
یک طرف دوگانه شماری که نیازمند اثبات است, بدیهی در نظر گرفته شده است.
\solution*{پاسخ صحیح}

سوال را با دوگانه شماری حل می کنیم:\\
   فرض کنید $ P$ بیانگر تعداد راه های انتخاب یک کمیته از بین $n$ کاندیدا است به طوری که یک فرد رئیس کمیته و یک نفر معاون باشند و رئیس و معاون می توانند یک نفر باشند. شمارش این راه ها به 2 روش امکان پذیر است.
    \begin{enumerate}
        \item 
        با فرض یکسان بودن رئیس و معاون، رئیس را انتخاب کرده و تصمیم می‌گیریم که بقیه افراد حضور داشته باشند یا خیر و حالات به دست آمده را جمع می‌کنیم با حالاتی که رئیس و معاون متمایز را انتخاب کردیم و در مورد حضور یا عدم حضور بقیه افراد تصمیم گرفتیم:
        \[P = n\times{2^{n-1}} + n\times(n-1)\times{2^{n-2}} = n\times(n+1)\times{2^{n-2}}\]
        \item
        ابتدا این که چه اعضایی کمیته و رئیس و معاون را تشکیل دهند را انتخاب می‌کنیم که این تعداد می‌تواند هر عددی باشد، سپس رئیس و معاون یکسان یا متمایز را از بین آن ها انتخاب می‌کنیم:
        \[P = \displaystyle\sum_{k=0}^{n} {\binom{n}{k}(k(k-1)+k)} = \displaystyle\sum_{k=0}^{n} {{k^2}\binom{n}{k}}\]
    \end{enumerate}
    از تساوی 2 حالت فوق حکم مساله اثبات می‌شود:
    \[\displaystyle\sum_{k=0}^{n} {{k^2}\binom{n}{k}} = n\times(n+1)\times{2^{n-2}}\]
    \\ \\
    
%% question 3

\question
۶۰ دانشجو در کلاس ریاضیات گسسته حضور دارند.در میان هر ۱۰ نفر از این کلاس , حداقل ۳ نفر نمره مبانی یکسانی دارند. ثابت کنید در این کلاس ۱۵ نفر وجود دارند که نمره مبانی آن ها یکسان است.

\solution

در نظر میگیریم حداکثر تعداد تکرار از یک نمره ۱۴ عدد است که در این صورت حداقل به ۵ نمره متفاوت نیاز است . 
در این صورت باز میتوان گروه ۱۰ تایی را از دانش آموزان انتخاب کرد که حداکثر دو نفر نمره یکسان داشته باشند . پس فرض اولیه غلط بوده و مشخص میشود که لااقل از یکی از نمرات وجود دارد که ۱۵ دانش آموز یا بیشتر 
آن نمره را دارند.
 


\notes

\Nnote{n7}
در پاسخ از برهان خلف استفاده شده ولی از آن نام برده نشده است و باید توجه کنیم فرض خلف را حتما بیان کنیم.


\Nnote{n8}
باید اصل لانه   کبوتری که از آن استفاده کرده است را نام میبرد و نحوه استفاده از آن مشخص شود.


\Tnote{n9}
پاسخ کامل نیست. پاسخ درست و کامل در پایین آمده است.



\solution*{پاسخ صحیح}

از برهان خلف استفاده می‌کنیم. فرض خلف : فرض کنید در این کلاس هیچ ۱۵ نفری وجود نداشته باشند که نمره‌ی مبانی آن‌ها یکسان باشد.
در این صورت حداکثر ۱۴ نفر وجود دارند که نمره‌ی یکسان داشته باشند. بنابراین  طبق اصل لانه کبوتری حداقل به اندازه‌ی سقف
$\frac{60}{14}$
یعنی ۵ نمره‌ی متفاوت در کلاس وجود دارد. مسئله را به دو حالت تقسیم می‌کنیم ؛
\begin{enumerate}
    \item 
    اگر پنج نمره‌ی متمایز وجود داشته باشند که از هر کدام ۲ عضو
    (دو نفر در کلاس که آن نمره را دارند)
    وجود داشته باشد؛ در این صورت از هر کدام از این نمرات دو عضو را درنظر گرفته و به مجموعه‌ای ۱۰ عضوی
    می‌رسیم که هیچ سه نفری در آن نمره‌ی یکسان ندارند که این خلاف فرض مسئله است و به تناقض رسیدیم.
    پس فرض خلف رد شده و حداقل ۱۵ نفر وجود دارند که نمره‌ی یکسانی داشته باشند.
    
    \item 
    اگر پنج نمره‌ی متمایز، هرکدام دارای حداقل دو عضو وجود نداشته باشند؛
    در این صورت 
    $k \le 4$
    نمره‌ی متمایز با بیش از یک عضو داریم
    (مجموعه‌ی این نمرات را A بنامیم)
    که با توجه به فرض خلف، حداکثر تعداد
    $14 \times k$
    عضو را پوشش می‌دهند. بنابراین حداقل 
    $60 - 14k$
    نفر باقی مانده که هیچ دو تایی نمی‌توانند دارای نمره‌ی یکسان باشند
    (در غیر این صورت تعداد نمره‌های متمایز دارای بیشتر مساوی ۲ عضو به حداقل k+1 می‌رسد).
    بنابراین هر یک از این اعضا دارای نمره‌ای متمایز است
    (مجموعه‌ی این اعضا را B بنامیم).
    می‌توان با انتخاب دو عضو از هر نمره‌ی مجموعه‌ی
    A
    و تمام اعضای مجموعه‌‌ی
    B
    به مجموعه‌ای متشکل از
    $N = 2k + 60 - 14k = 60 - 12k$
    عضو رسید که
    $k \le 4 \Rightarrow N \ge 12$
    و هیچ سه عضوی در آن دارای نمره‌ی یکسان نیستند.
    هر ده عضوی از این مجموعه انتخاب شود، نقض فرض مسئله است و به تناقض رسیدیم.
    پس فرض خلف رد شده و حداقل ۱۵ نفر وجود دارند که نمره‌ی یکسانی داشته باشند.
\end{enumerate}


%% question 4

\question
ضریب عبارت $ x ^ {12}$
در بسط عبارت
$(1-4x)^{-5}$
را بیابید.

\solution
طبق بسط دوجمله ای داریم:    
    \begin{align*}
    \frac{1}{(1-4x)^5}=\sum\limits_{k=0}^{\infty} \binom{k+4}{k}  4 ^ k  x^ k              
    \end{align*}
    
    جمله ۱۲ ام دنباله 
    $ a_n$
  ضریب 
  $ x ^ {12} $
  است.
$ {\textcolor{byzantine}{(E.XI)}}  $

\begin{align*}
  \longrightarrow a_{12} = \binom{16}{12} 4 ^ {12}
\end{align*}



\notes

\Nnote{n10}
بهتر است اصل بسط دوجمله ای هم نوشته شود.

\Enote{n11}
قبل از استفاده از متغیر باید آن را تعریف کرد. تعریف دنباله $a_n$ ضروری است. 

\solution*{پاسخ صحیح}

طبق جدول
  Functions Generating Useful
از کتاب Rosen
که استاد نیز به آن اشاره کردند داریم:\\
$$(1 - x) ^ {-n} = \sum_{k = 0}^{\infty} \binom{n + k - 1}{k} x^{k} $$
  بنابراین در این سوال داریم:
$$(1 - 4x) ^ {-5} =
\sum_{k = 0}^{\infty} \binom{5 + k - 1}{k} (4x)^k $$

جمله $x^{12}$ به ازای مقدار $k = 12$ ساخته می‌شود. بنابراین جواب برابر خواهد بود با: 
$$\binom{16}{12} (4)^{12}$$

%% question 5

\question
چند عدد طبیعی حداکثر ۹ رقمی وجود دارد که مجموع ارقام آن برابر با ۳۲ باشد؟\\ \\
\solution
  سوال را با اصل شمول و عدم شمول حل می کنیم: 
   %\[|A_1\cup A_2\cup A_3\cup A_4\cup A_5\cup A_6\cup A_7\cup A_8\cup A_9|=|A_1|+|A_1\cup A_2|+|A_1\cup A_2\cup A_3|+|A_1\cup A_2\cup A_3\cup A_4|\]
   
    \[|A_1\cup A_2\cup... \cup A_9|=\binom{9}{1}|A_1|+\binom{9}{2}|A_1\cap A_2|+...+\binom{9}{9}|A_1\cap A_2\cap...\cap A_9|\]\\
    حال مقدار عبارت ها را حساب می کنیم:\\
    \[|A_1|=\binom{30}{8}\]
    \[|A_1\cap A_2|=\binom{20}{8}\]
    \[|A_1\cap A_2\cap A_3|=\binom{10}{8}\]
    برای بقیه جمله ها جواب برابر ۰ است.\\
    حال از اصل متمم برای به دست آوردن جواب نهایی استفاده می کنیم:\\
        -کل حالات:
     \[\binom{40}{8}\]\\
     - حالات مطلوب:
     \[\binom{40}{8}- \binom{9}{1}\binom{30}{8}+\binom{9}{2}\binom{20}{8}-\binom{9}{3}\binom{10}{8}\]
     
\notes
\Enote{n12}
تعریف متغیر های $A_i$ ضروری است, چون در غیر این صورت منظور از بقیه استدلال ها به هیج وجه مشخض نیست.
\Enote{n13}
اثبات و یا در صورت وضوح, اشاره به تقارن میان مجموعه ها برای استفاده از اصل شمول و عدم شمول به این شکل ضروری است.
\solution*{پاسخ صحیح}

رقم $i$ ام این عدد را با $x_i$ نشان می دهیم, بنابراین به دنبال یافتن تعداد جواب های صحیح معادله زیر هستیم:
    \[\sum\limits_1^9 x_i=32\]
    \[\forall i \in [1,9] ,i \in N : x_i\leq 9\]
    %$\sum\limits_1^9 x_i=32$
    %$\exists i \in [1,9] ,i \in N : x_i\geq 10$
  تعداد جواب های صحیح این معادله را به کمک اصل متمم پیدا می کنیم:\\
     -کل حالات:تعداد جواب های صحیح نامنفی معادله  $\sum\limits_1^9 x_i=32 $ .این یک معادله سیاله است و تعداد جواب های صحیح آن برابر است با:
     
     \[\binom{40}{8}\]\\
    -حالات نامطلوب:تعداد جواب های صحیح نامنفی معادله $\sum\limits_1^9 x_i=32$  
     به طوری که:\\
     ($\exists i \in [1,9] ,i \in N : x_i\geq 10$)\\
     حال اگر مجموعه حالت هایی که در آن $x_i\geq 10 $ است را با $A_i$نشان دهیم, کافی است تعداد اعضای اجتماع این مجموعه ها را بیابیم\\ 
     طبق اصل شمول و عدم شمول و با توجه به تقارن میان $A_i$ ها داریم: 
     \[|A_1\cup A_2\cup... \cup A_9|=\binom{9}{1}|A_1|+\binom{9}{2}|A_1\cap A_2|+...+\binom{9}{9}|A_1\cap A_2\cap...\cap A_9|\]
   % \[|\bigcup\limits^9_{i=1}A_i|=\sum_{k=1}^{9}(-1)^{k+1}\binom{9}{k}|\bigcap\limits^k_{i=1}A_i|\]\\%
   برای محاسبه مقدار عبارت ها, در معادله سیاله متناظر, در صورتی که $x_i\geq 10$ بود قرار می دهیم $x_i=y_i+10 $ و در غیر این صورت قرار می دهیم $x_i=y_i$, حال اگر تعداد $i$ هایی را که به ازای آن ها $x_i\geq 10 $ است را با $k$ نشان بدهیم, حال به دنبال تعداد جواب های صحیح نامنفی معادله سیاله $\sum\limits_1^9 y_i=32-10k $ هستیم, که برابر است با: 
   \[\binom{40-10k}{8}\] 
    حال مقدار عبارت ها را حساب می کنیم:\\
    \[|A_1|=\binom{30}{8}\hspace{2cm} (k=1)\]
    \[|A_1\cap A_2|=\binom{20}{8}\hspace{2cm} (k=2)\]
    \[|A_1\cap A_2\cap A_3|=\binom{10}{8}\hspace{2cm} (k=3)\]
    برای بقیه جمله ها جواب برابر ۰ است.\\
    پس کل حالات نامطلوب برابر است با:
    \[\binom{9}{1}\binom{30}{8}-\binom{9}{2}\binom{20}{8}+\binom{9}{3}\binom{10}{8}\]
     - حالات مطلوب: طبق اصل متمم برابر است با:
     \[\binom{40}{8}- \binom{9}{1}\binom{30}{8}+\binom{9}{2}\binom{20}{8}-\binom{9}{3}\binom{10}{8}\]
    
    
%% question 6

\question 
   با استفاده از توابع مولد نشان دهید تعداد روش های انتخاب ۴ عضو دو به دو  نامتوالی از مجموعه اعداد n
			 ,...,1,2,3
			 برابر با 
			 انتخاب 4 از n-3 است.
			 
\solution
یک زیرمجموعه از این نوع مثلا {۱و۳و۷و۱۰} را انتخاب و نابرابری های اکید 
    $ \\ 0 < 1 < 3 < 7 < 10 < n+1 \\ $ 
   را در نظر میگیریم. و بررسی میکنیم چند عدد صحیح بین هر دو عدد متوالی از این اعداد وجود دارند. در اینجا ۰ و ۱ و۳ و
   \lr{2}
    و n-۱۰ 
   را به دست می آوریم: ۰ زیرا عددی صحیح بین
    \lr{0}
   و ۱ وجود ندارد و ۱ زیرا تنها عدد ۲ بین ۱ و۳ وجود دارد و
   \lr{3} 
   زیرا اعداد صحیح ۴ و۵ و۶ بین ۳ و۷ وجود دارند و ... .
   مجموع این ۵ عدد صحیح برابر
  \lr{ $ 0 + 1 + 3 + 2 + n-10 = n-4 $} 
     است.
     $ {\textcolor{byzantine}{(E.XIV)}}  $
     
    پس تابع مولد زیر را داریم.
    
   \begin{align*}
   G(x) = ( 1 + x^2 + x^3 + ...)^2 (x + x^2 + x^3 + ...)^3 = ( \sum_{k = 0}^{\infty} x^k)^2 ( \sum_{k = 0}^{\infty} x^{k+1})^3 =\\ \frac{1}{(1-x)^2} . (\frac{x}{1-x})^3 = \frac{x^3 }{(1-x)^5} = x^2 (1-x)^{-5} = x^3 \sum_{k=0}^{\infty} \binom{k+4}{k} x ^ k =\\ \sum_{k=0}^{\infty} \binom{k+4}{k} x^{k+3} = {\textcolor{byzantine}{(E.XV)}} \hspace{0.5 cm} \sum_{k=0}^{\infty}\binom{k+1}{k-3} x^k
\end{align*}    
      
      به دنبال ضریب
      $ x ^ {n-4 } $
      میگشتیم پس 
      $ k = n-4 $
      و جواب نهایی برابر است با
      \lr{$\binom{n-3}{n-7} = \binom{n-3}{4}$} \\ \\ \\ 
  
 
      
\notes

\Enote{n14}
  مثال زدن باید به صورتی باشد که حذف آن اختلالی در فهم جواب ایجاد نکند . در اینجا اگر مثال پاراگراف اول را حذف کنیم مشخص نیست تابع مولد برچه اساسی نوشته شده است. پس باید توضیحی درمورد تابع مولد و جمله ای که به دنبال ضریب آن هستیم بدهیم .
		
\Enote{n15}
     نیاز هست که کاملا گفته شود چه تغییر متغیری انجام میشود . در اینجا تغییر متغیر $ k \rightarrow {k+3} $   را داریم.  همیشه به هنگام تغییر متغیر توجه کنیم ممکن است کران ها تغییر کنند. در اینجا کران پایین از صفر به سه میرود. 
	 صورت اصلاح شده:
	 $ \sum_{k=3}^{\infty}\binom{k+1}{k-3} x^k $
	
\Nnote{n16}		
	  در طی پاسخ به سوال خوب است دقت کنیم همه ی اعداد را یا فارسی یا انگلیسی 
	  بنویسیم.
	  
	  
\solution
  تابع مولد فاصله از مبدا:
   \begin{align*}
   G(x)=(1+x+x^2+...)(x+x^2+x^3+...)^3
   \end{align*}
   در مجموع n-4 عدد داریم . توان های x باید بین مبدا و مقصد باشند پس باید توانی از x را که کوچک تر یا مساوی n-4 هستند را بیابیم:
   
   \begin{align*}
   G(x)= \frac{x^3}{(1-x)^4} = x^3 (1-x)^{-4} = x^3 \sum_{k=0}^{\infty}\binom{k+3}{3} x^k \\ 
   \longrightarrow \sum_{k=0}^{\infty}\binom{x+k}{k} = \frac{1}{(1+x)^{k+1}} 
   \longrightarrow k+3 \le n-4 \rightarrow k \le n-7
   \end{align*}
   
   \begin{align*}
     \binom{n+1}{r+1} = \sum_{k=r}^{n} \binom{k}{r} \hspace{2cm} (1) \\
  \end{align*}
  
   مجموع حالات: 
   \begin{align*}
   \longrightarrow \sum_{k=0}^{n-7}\binom{k+3}{3} \xrightarrow{(1)}  \binom{n-7+4}{4} = \binom{n-3}{4}   \hspace{2cm} 
   \end{align*}
   
   
   
 \notes
 
 \Enote{n17}
   به هنگام جایگذاری در فرمول باید جایگذاری ها واضح باشد. در این مثال در فرمول (۱) کران پایین از r هست ولی در قسمتی که از آن استفاده شده کران پایین از ۰ است. همین مطلب گویای آن است که به توضیحات بیشتری نیاز هست. 
		
		عبارت زیر صورت کامل شده این نکته است:
		\begin{align*}
		\longrightarrow \sum_{k=0}^{n-7} \binom{k+3}{k} = \sum_{k=3}^{k-4} \binom{k}{k-3} = \sum_{k=3}^{n-4} \binom{k}{3} \xrightarrow[{r \rightarrow 3},{ n \rightarrow {n-4} }]{(1)} \binom{n-3}{4}
\end{align*}	


\solution*{پاسخ صحیح}

   تعداد عضو‌های  انتخاب نشده کوچکتر از عضو اول انتخاب شده را 
                $x_1$،
                عضوهای انتخاب نشده بین عضو اول و دوم انتخاب شده را
                $x_2$،
                عضو‌های انتخاب نشده بین عضو دوم و سوم انتخاب شده را
                $x_3$،
                عضو‌های انتخاب نشده بین عضو سوم و چهارم انتخاب شده را
                $x_4$ و
                عضوهای انتخاب نشده بزرگ‌تر از چهارمین عضو انتخاب شده را 
                $x_5$
                می‌گیریم. کافی است تعداد جواب‌های صحیح نامنفی معادله زیر را با شرایط 
                $x_1, x_5 \geq 0 \; x_2 x_3, x_4 \geq 1$
                بشماریم
               
                $$x_1 + x_2 + x_3 + x_4 + x_5 = n - 4$$
                که برابر است با ضریب
                $x^{n - 4}$
                در عبارت:\\
              
                $$(1 + x + x^2 + ...)(x + x^2 + x^3 + ...)(x + x^2 + x^3 + ...)(x + x^2 + x^3 + ...)(1 + x + x^2 + ...) = \frac{x^3}{(1 - x)^5}$$
                
                 بنابراین کافی است ضریب $x^{n - 7}$ را در بسط
                  $(1 - x)^{-5}$
         بشماریم . 
               
               
                   طبق جدول  
                  Functions Generating Useful
		از کتاب Rosen
		که استاد نیز به آن اشاره کردند داریم:\\
       
		    $$(1 - x) ^ {-n} = \sum_{k = 0}^{\infty} \binom{n + k - 1}{k} x^{k} $$

	    بنابراین در این سوال داریم:
                
                    $$(1 - x)^{-5} = \sum_{k = 0}^{\infty} \binom{5 + k - 1}{k}x^k
                    = \sum_{k = 0}^{\infty} \binom{k + 4}{k}x^k
                    = \sum_{k = 0}^{\infty} \binom{k + 4}{4}x^k$$
             
                $x^{n - 7}$ به ازای $k = n -7$ ساخته می‌شود. بنابراین جواب برابر است با:\\
               
                    $$\binom{n - 3}{4}$$
        

%% question 7
\question
 اتحاد زیر را ثابت کنید.
     \begin{align*}
     \sum_{i=0}^{n} i{\binom{n}{i}}^2 = \frac{n}{2} \binom{2n}{n}
     \end{align*}
     
\solution
\begin{align*}
A=\sum_{i=0}^{n} i{\binom{n}{i}}^2 \xrightarrow{\times 2} 2A = \sum_{i=0}^{n} i{\binom{n}{i}}^2 + \sum_{i=0}^{n} i{\binom{n}{i}}^2 \xrightarrow{ j=n-i } \\ 
2A = \sum_{i=0}^{n} i{\binom{n}{i}}^2 + \sum_{j=0}^{n} (n-j){\binom{n}{n-j}}^2 \rightarrow 2A = \sum_{i=0}^{n} i{\binom{n}{i}}^2 + \sum_{j=0}^{n} (n-j){\binom{n}{j}}^2 = \\
 \sum_{i=0}^{n} (i + (n-i)) {\binom{n}{i}}^2 \rightarrow 2A = n \sum_{i=0}^{n} {\binom{n}{i}}^2 \rightarrow 2A = n \sum_{i=0}^{n} \binom{n}{i} \binom{n}{n-i}
\end{align*}
بنابراین میدانیم که:
  \begin{center}
  $  2A = n\binom{2n}{n} \rightarrow A= \frac{n}{2} \binom{2n}{n} $
  \end{center}


\notes

\Nnote{n18}  باید فرمول و اتحاد های مورد استفاده و رفرنس معتبر آن ‌ذکر شود . به عنوان رفرنس اسم اتحاد هم کافی است. 
       	 
 
 
 \solution*{پاسخ صحیح}
 طبق اتحاد واندرموند داریم:

$$ \sum_{i=0}^{k} \binom{m}{i} \binom{n}{k-i} = \binom {m+n}{k} \hspace{2cm} (1) $$
\begin{align*}
A=\sum_{i=0}^{n} i{\binom{n}{i}}^2 \xrightarrow{\times 2} 2A = \sum_{i=0}^{n} i{\binom{n}{i}}^2 + \sum_{i=0}^{n} i{\binom{n}{i}}^2 \xrightarrow{ j=n-i } \\ 
2A = \sum_{i=0}^{n} i{\binom{n}{i}}^2 + \sum_{j=0}^{n} (n-j){\binom{n}{n-j}}^2 \rightarrow 2A = \sum_{i=0}^{n} i{\binom{n}{i}}^2 + \sum_{j=0}^{n} (n-j){\binom{n}{j}}^2 = \\
 \sum_{i=0}^{n} (i + (n-i)) {\binom{n}{i}}^2 \rightarrow 2A = n \sum_{i=0}^{n} {\binom{n}{i}}^2 \rightarrow 2A = n \sum_{i=0}^{n} \binom{n}{i} \binom{n}{n-i} \xrightarrow[m=n=k]{(1)} \\
 2A= n \binom{2n}{n} \rightarrow A= \frac{n}{2} \binom{2n}{n}
\end{align*}


\end{document}